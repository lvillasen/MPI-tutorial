\documentclass[letter]{jpconf}

\usepackage{xcolor}


    
    
    
\usepackage{listings}
\usepackage{float}
\usepackage{graphicx}
\usepackage{caption}
\DeclareCaptionFont{white}{\color{white}}
\DeclareCaptionFormat{listing}{%
\parbox{\textwidth}{\colorbox{gray}{\parbox{\textwidth}{#1#2#3}}\vskip-4pt}}
\captionsetup[lstlisting]{format=listing,labelfont=white,textfont=white}
\lstset{frame=lrb,xleftmargin=\fboxsep,xrightmargin=-\fboxsep}

\renewcommand{\abstractname}{Resumen}
\renewcommand{\lstlistingname}{Listado}
\renewcommand{\tablename}{Tabla}

\makeatletter
\newcommand{\verbatimfont}[1]{\def\verbatim@font{#1}}%
\makeatother
\usepackage{titlesec}

\titleformat{\chapter}
       {\normalfont\fontfamily{phv}\fontsize{18}{21}\bfseries}{\thechapter}{1em}{}

\titleformat{\section}
       {\normalfont\fontfamily{phv}\fontsize{14}{19}\bfseries}{\thesection}{1em}{}

\titleformat{\subsection}
       {\normalfont\fontfamily{phv}\fontsize{12}{17}\bfseries\itshape}{\thesubsection}{1em}{}


\begin{document}
 \vspace*{-45mm}
\title{Ejercicios sobre el uso de MPI en el Laboratorio Nacional de Superc\'omputo del Sureste de M\'exico}
\author{Luis M. Villase\~nor Cendejas}

\address{Benem\'erita Universidad Aut\'onoma de Puebla, Puebla, Mexico}

\ead{lvillasen@gmail.com}
\author{\textbf{2 de diciembre de 2016}}
\section{Introducci\'on}
\vspace{1\baselineskip}

Este documento contiene una lista de ejercicios sobre MPI~\cite{mpi}. Estos ejercicios se proponen para complementar el tutorial sobre MPI que est\'a disponible en~\cite{github,LNS}.
De acuerdo con este tutorial, que explica el uso b\'asico de MPI en C, Fortran, Julia y Python, el lenguaje de programaci\'on para hacer estos ejercicios puede ser cualquiera de estos cuatro.

\section{Ejercicios}
\vspace{1\baselineskip}

\begin{itemize} 
\item 

Elaborar un programa  usando MPI para hacer el producto escalar de dos vectores de forma paralela. El producto escalar de dos vectores se obtiene haciendo la suma de los productos entre los elementos de los vectores.

\item 

Elaborar un programa  usando MPI para hacer  la multiplicaci�n paralela de una matriz (NxN) con un vector (Nx1) para obtener un vector (Nx1).

\item 

Elaborar un programa  usando MPI para hacer  la multiplicaci�n paralela de una matriz (LxM) por una matriz (MxN) para obtener una matriz (LxN).

\item 

Elaborar un programa usando MPI  que implemente un "ping-pong", en el que
el proceso ra\'iz env\'ia un n\'umero a un proceso  y este \'ultimo lo devuelve inmediatamente al proceso ra\'iz aumentado en uno.

\item 

Elaborar un programa usando MPI  que implemente el juego de "papa caliente", en el que
un proceso aleatorio env\'ia un n\'umero a otro proceso escogido al azar,  y este \'ultimo a su vez lo manda a otro proceso, 
aumentado en uno y tambi\'en escogido al azar. Cuente el n\'umero de env\'ios antes de que el n\'umero regrese al proceso inicial.


\item 


Elaborar un programa usando MPI  que determine si el
n\'umero de unos en una secuencia binaria grande, almacenada en un arreglo, es par o
impar.


\item 


Elaborar un programa usando MPI  para contar los n\'umeros primos menores que un n\'umero dado.
Como gu\'ia, en el Listado~\ref{primo_py}  se muestra un programa secuencial en Python, 
y en el Listado~\ref{primo_jl}  se muestra otro programa secuencial en Julia, para este prop\'osito.


\item 


Elaborar un programa usando MPI  para  paralelizar el fractal de Mandelbrot~\cite{mandel_ref}.  En el Listado~\ref{mandel_jl} se muestra un programa secuencial 
escrito en Julia para este prop\'osito.  En la Figura~\ref{mandel_fig} se muestra el fractal de Mandelbrot obtenido con este programa. Puede usar el lenguaje de su preferencia.



\item 


Elaborar un programa usando MPI  para  implementar el "Juego de la Vida"~\cite{vida_ref}.
Como gu\'ia,  el Listado~\ref{vida_jl}  muestra un programa secuencial escrito en Julia para este prop\'osito. Este c\'odigo produce la Figura~\ref{vida_fig}.
Puede usar el lenguaje de su preferencia.

\item 

Elaborar un programa usando MPI  para procesar una malla bidimensional X de
$1000 \times 1000$ puntos, actualiz\'andola de acuerdo al siguiente esquema iterativo:
\newline

\centerline
{$X^{t+1}_{i,j}$ = $(X_{i + 1, j + 1}^t + X^t_{i-1,j-1} + X^t_{i+1,j-1} + X^t_{i-1,j+1})/4$}

\vspace{1\baselineskip}

Para resolver el problema en los bordes de la malla considere condiciones peri\'odicas. Suponga que los valores iniciales son $X_i=10$ en los bordes de la malla y $X_i=0$  
en los puntos interiores. Establecer un criterio de convergencia. 

\end{itemize} 


\begin{lstlisting}[float,floatplacement=H,label=primo_py,caption=Programa secuencial  para calcular n\'umeros primos en Python.]
import numpy as np
N1 =800000
N2 = 850000
if (N1%2 == 0):
    N1+=1
if (N2%2 == 0):
    N2-=1
n= N1
Num_primos=0
while (n<=N2):
    n1=3
    p=1
    while (n1<np.sqrt(n) and p ==1):
        if (n%n1 == 0):
            p=0
        n1+=2
    if (p==1):
        Num_primos+=1
    n+=2
print "Num_primos=",Num_primos
 \end{lstlisting}


\begin{lstlisting}[float,floatplacement=H,label=primo_jl,caption=Programa secuencial  para calcular n\'umeros primos en Julia.]
N1 =8000000
N2 = 8500000
if N1%2 == 0; N1+=1;end
if N2%2 == 0; N2-=1;end
n= N1
Num_primos=0
while n<=N2
    n1=3
    p=1
    while n1<sqrt(n) && p ==1
        if n%n1 == 0;p=0; end
        n1+=2
    end
    if p==1;Num_primos+=1 ;end
    n+=2
end
print("Num_primos=$Num_primos \t")
 \end{lstlisting}
 
 \begin{lstlisting}[float,floatplacement=H,label=mandel_jl,caption=Programa secuencial  para graficar el fractal de Mandelbrot en Julia.]
 using PyCall
@pyimport matplotlib.pyplot as plt
const N=800
function mandelbrot(x0,y0,side,L=30,R=3.)
    m = zeros(N,N)
    delta = side/N
    for i=1:N, j=1:N
        c = x0+delta*i+(y0+delta*j)*im
        z, h = 0+0*im, 0
        while (h<L) && (abs(z)<R)
            z = z^2+c
            h+=1
        end
        m[j,i]=h
    end
    return m
end
n=2.6
m=mandelbrot(-n/1.5,-n/1.9, n)
plt.imshow(m)
plt.show()
 \end{lstlisting}


\begin{lstlisting}[float,floatplacement=H,label=vida_jl,caption=Programa secuencial del "Juego de la Vida" en Julia.]
#Usage julia GameOfLife.py n n_iter m
# n is the board size 
# n_iter the initial number of iterations
# m is 1 to plot the board or 0 for no graphical output 
using PyCall
@pyimport matplotlib.pyplot as plt
const n = int(ARGS[1])
n_iter = int(ARGS[2])
const m = int(ARGS[3])
function life_game(n,M,n_iter)
    for k=1:n_iter
        C = copy(M) # copy current life grid
        for i=1:n, j=1:n
            if (i==1) i_left=n else i_left=i-1 end
            if(i==n) i_right=1; 
            else i_right=i+1;end
            if(j==1) j_down=n; else j_down=j-1;end
            if(j==n) j_up=1; 
            else j_up=j+1;end
            count = C[i_left,j] + C[i_left,j_up] + C[i_left,j_down] +
             C[i,j_down] + C[i,j_up] + C[i_right,j] + 
             C[i_right,j_up] + C[i_right,j_down] 
            if C[i, j]==1
                if count < 2 || count > 3
            M[i, j] = 0 # living cells with <2 or >3 neighbors die
                end
            elseif count == 3
                M[i, j] = 1 # dead cells with 3 neighbors are born
            end; end; end
    return M
end
M0=rand(0:1,n,n) # Initialize with a random pattern
M=life_game(n,M0,n_iter)
print(sum(M)," live cells after ",n_iter," iterations\n")
if m != 0
    fig = plt.figure(figsize=(12,12))
    plt.gray()
    while m != 0
        M1=abs(1-M)
        plt.imshow(M1,  interpolation="none")
        plt.title("Conway's Game of Life, $n_iter Iterations")
        plt.pause(.01) 
        #print(" k=",n_iter," live cells=",sum(M),"\n")
        M=life_game(n,M,1) 
        n_iter=n_iter+1; end; end
\end{lstlisting}


\begin{figure}
\includegraphics[width=\linewidth]{/Users/luis/Dropbox/2016_Prod_Academica/Tutorial_MPI_LNS/mandel.png}
  \caption{Gr\'afica del fractal de Mandelbrot obtenida con el programa mostrado en  el Listado~\ref{mandel_jl}.}
  \label{mandel_fig}
\end{figure}


\begin{figure}
\includegraphics[width=\linewidth]{/Users/luis/Dropbox/2016_Prod_Academica/Tutorial_MPI_LNS/vida.png}
  \caption{Gr\'afica del "Juego de la Vida" obtenida con el programa mostrado en el Listado~\ref{vida_jl}.}
  \label{vida_fig}
\end{figure}

\section*{Bibliograf\'ia}
\begin{thebibliography}{9}
\bibitem{mpi}http://dl.acm.org/citation.cfm?id=898758
\bibitem{github} https://github.com/lvillasen/TutorialMPI
\bibitem{LNS} http://www.lns.org.mx/node/131
\bibitem{mandel_ref}https://en.wikipedia.org/wiki/Mandelbrot\_set
\bibitem{vida_ref}https://en.wikipedia.org/wiki/Conway's\_Game\_of\_Life

\end{thebibliography}

\end{document}


